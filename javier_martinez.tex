\documentclass{simplecv}
\usepackage[latin1]{inputenc}
\usepackage[spanish]{babel}
\usepackage{hyperref}
\usepackage{url}

\begin{document}

\leftheader{+34 682 39 81 69\\
\texttt{\small martinez.javier@gmail.com}}

\rightheader{Carrer de Hurtado 34, Barcelona, Spain\\
\texttt{\small \url{http://martinezjavier.blogspot.com}}
}

\title{Javier Mart�nez Canillas}

\maketitle

\section{Summary}

I'm an open-source enthusiast software engineer with experience working on different layers of the software stack.

\section{Interests}

Software development, Linux kernel development, Linux device drivers.

\section{Education}

\begin{topic}

\item[2010 - 2011] {\bf  M.S. in High Performance Computing} - Universitat Aut�noma de Barcelona, Bellaterra, Spain. 

{\bf Thesis:} \emph{Including the Workload Effect in the Parallel Program Signature}. 

Advisor: Prof. Emilio Luque.

\item[2002 - 2008] {\bf  B.S. in Computer Engineering} - Catholic University "Nuestra Se�ora de la Asunci�n", Asunci�n, Paraguay. 

{\bf Thesis:} \emph{A new approach to solve Multi-Objective Evolutionary Optimization Problems using Linear Genetic Programming}. 

Advisor: Prof. Benjam�n Bar�n.

\end{topic}

\section{Awards and Honors}

PIF-UAB predoctoral scholarship for Master of Science studies at Universitat Aut�noma de Barcelona.

\section{Professional Experience}

\begin{topic}

\item[September 2011 - Present] \emph{Linux Kernel Engineer} - ISEE (\url{http://www.iseebcn.com}): Embedded systems manufacturer.

Linux Kernel and bootloader (X-loader and U-boot) porting and device drivers development for custom system-on-chip boards.

Platform: Linux on ARM TI OMAP3 (Cortex-A8).

Technologies: C, Kernel Build System (Kbuild), Git.

Repository: \url{http://git.igep.es/}

\item[October 2010 - August 2011] \emph{M.Sc. Student and Researcher} - Computer Architecture and Operating System department, Universitat Aut�noma de Barcelona.

Design and implementation of a parallel application performance analysis and prediction tool.

Doing research to improve the tool accuracy.

Plataform: Linux clusters

Technologies: C/C++, MPI, GNU Build System (Autotools), GTK.

\item[November 2009 - September 2010] \emph{Software Engineer} - TDI S.A. (\url{http://tdi.com.py}): Software factory.

Designed and developed an enterprise resource planning for a distribution company and a portal solution for a non-governmental organization.

Platform: Java Enterprise Edition 6 and Liferary.

Technologies: Java Enterprise Edition development consulting: Glassfish, Java Server Faces 2.0, Enterprise Java Beans 3.1, Java Persistence API 2.0, JAX-WS, JAX-RS, Facelets, JasperReports, Jboss, Liferay, Jboss Seam, Facelets.

%\item[July 2009 - September 2010] Professor and Researcher  - Computer Engineering, Polytechnic Faculty, National University of Asunci�n.

%\begin{itemize}

%\item Updated the Linux Device Drivers 3 book examples to compile and run with newer kernels. So students can develop their own virtual device drivers based in those examples:

%Repository: \url{https://github.com/martinezjavier/ldd3}

%\item Developed a sniffer as a kernel module using the kernel's protocol handler (\emph{struct packet\_type}) kernel infrastructure to help students understand packet reception.

%Repository: \url{https://github.com/martinezjavier/ksniffer}

%\item Developed a firewall as a kernel module using netfilter register hooks API (\emph{struct nf\_hook\_ops}) to help students understand data structures used by the kernel to store frames (\emph{struct sk\_buff}) and headers (\emph{struct iphdr, struct tcphdr, struct ethhdr}).

%Repository: \url{https://github.com/martinezjavier/kfirewall}

%\end{itemize}

\item [December 2008 - October 2009] \emph{Software Engineer} - Tabacos del Paraguay S.A. (\url{http://www.tabacosdelparaguay.com}): Cigarettes distribution company.

Designed and built a data warehouse and geographic information system.

Platform: Microsoft business intelligence and Pentaho.

Technologies: Pentaho Data Integration (Kettle), Pentaho Analysis (Mondrian), Pentaho Reporting, Pentaho BI Server, MS Integration Services, MS Analysis Services, MapServer, GeoServer, Postgresql, Postgis, MySQL Spatial Extensions, OpenLayers, Mapfish, Google Maps API. 

\item [May 2008 - November 2008] \emph{Software Engineer} - Telecom Personal Paraguay (\url{http://www.personal.com.py}): Cell phone and internet service provider.

Designed and developed an enterprise mobile application suite and location based services.

Platform: Microsoft .Net and Gemalto.

Technologies: j2me, Gemalto SAT, $C\sharp$, ASP .Net, jQuery, Windows Services, MS SQL Server, Google Maps API, jQuery.

\item [March 2005 - May 2008] \emph{Developer and System Administrator} - Simeic (\url{http://www.simeic.com}): Consulting company.

Enterprise resource planning software development and Linux system administration.

Platform: Linux and Mono.

Technologies: $C\sharp$, GTK, PHP, Mysql, Apache, Bind, Samba, Postfix, Squid, Jabber, Iptables/Netfilter.

\end{topic}

\section{Technical skills}

\begin{itemize}

\item Languages: C, Bash, Python, Java, $C\sharp$, Javascript, PHP, SQL.

\item Frameworks: GTK, GNU toolchain, Kernel Build System (Kbuild).

\item Operating Systems: Linux (Kernel development, porting and device drivers).

\end{itemize}

\section{Open source contributions}

Linux Kernel (\url{http://git.kernel.org})

\url{http://www.ohloh.net/accounts/martinezjavier}

%\section{Publications in Refereed Journals}

%\begin{thebibliography}{10}

%\end{thebibliography}

%\section{Publications in Refereed Conferences}

%\begin{thebibliography}{10}

%\footnotesize

%\bibitem{mar08b}
%R. S�nchez, J. Martinez, B. Bar�n. \emph{Macro-Economic Time-Series Forecasting Using Linear Genetic Programming}, $11^{th}$ Joint Conference on Information Sciences, Dec 15-20, 2008, China. 

%\bibitem{mar08a}
%Javier Martinez Canillas, Roberto S�nchez, Benjam�n Bar�n. \emph{Estimation Models Generation using Linear Genetic Programming}. CLEI Electronic Journal Volume 12 Number 3, December 2009. Regular Issue and Special Issue of Best Papers presented at CLEI 2008, Santa Fe, Argentina.

%\bibitem{mar11a}
%Martinez Canillas, J. and Wong, A. and Rexachs, D. and Luque, E. \emph{Predicting Parallel Applications Performance using Signatures: the Workload Effect}. Computer Systems and Applications (AICCSA), 2011 IEEE/ACS International Conference on. December, 2011, Sharm El-Sheikh, Egypt.

%\bibitem{mar11b}
%Martinez Canillas, J. and Wong, A. and Rexachs, D. and Luque, E. \emph{Including the Workload Effect in the Parallel Program Signature}. High Performance Computing and Communications (HPCC), 2011 IEEE International Conference on. September, 2011, Banff, Canada.

%\end{thebibliography}

%\section{References:}

%\begin{itemize}

%\item Phd. Emilio Luque, Professor, PhD advisor, Universitat Autonoma de Barcelona, Phone/Fax:(+595-21) 371-222. Email: emilio.luque@uab.es.

%\item Phd. Benjam�n Bar�n, Professor, PhD program director, Polytechnic Faculty, National University of Asunci�n, Phone/Fax:(+595-21) 371-222. Email: bbaran@cnc.una.py.

%\item Phd. Luca Cernuzzi, Dean, Science and Technology Faculty, Catholic University "Nuestra Se�ora de la Asunci�n" . Phone: (+595-21) 334.650. Fax: (+595-21) 310-072. Email: lcernuzz@uca.edu.py.

%\end{itemize}

%\section{Additional Information}

%I am an Engineer very passionated about computer science and technology, self-motivated and hard-worker. I love to design and develop computer programs, solve hard problems and learn new algorithms, techniques and tools.

\end{document}
